

Maße des Biegebalkens:
\[
l = 117 \text{ mm}, \quad b = 19.8 \text{ mm}, \quad h = 2.94 \text{ mm}
\]
Maximale Last:
\[
M = 2.609 \text{ kg}
\]

\subsection*{Berechnung des Widerstandsmoments}
\[
W_x = \frac{b h^2}{6}
\]
Einsetzen der Werte:
\[
W_x = \frac{19.8 \times 2.94^2}{6} = 28.52 \text{ mm}^3
\]

\subsection*{Berechnung des Biegemoments}
\[
M_b = F \times l
\]
\[
M_b = 2.609 \times 9.81 \times 117mm = 2994.53 \text{ Nmm}
\]

\subsection*{Berechnung der Spannung}
\[
\sigma = \frac{M_b}{W_x}
\]
\[
\sigma = \frac{2994.53}{28.52} = 104.99 \text{ N/mm}^2
\]

\subsection*{Berechnung der Dehnung}
\[
\varepsilon = \frac{\sigma}{E}
\]
Mit \(E = 210000 \text{ N/mm}^2\):
\[
\varepsilon = \frac{104.99}{210000} = 0.499\times 10^{-3}
\]

\subsection*{Berechnung der Brückenausgabe}
\[
\frac{U_M}{U_B} = \frac{n}{4} \times k \times \varepsilon
\]
Mit \( n = 2 \), \( k = 2.01 \):
\[
\frac{U_M}{U_B} = \frac{2}{4} \times 2.01 \times 0.499 \times 10^{-3}
\]
\[
\frac{U_M}{U_B} = 0.000502 \text{ V/V} = 0.5 \text{ mV/V}
\]