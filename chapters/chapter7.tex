\chapter{Fazit}
Die Arbeit am durchgeführten Projekt stellte für uns eine neue Erfahrung dar, die uns wertvolle Einblicke in die Bereiche der Elektrotechnik und Mechanik ermöglichte.
Wir konnten interessante, praktische Arbeiten verrichten und viel über die Planung von Projekten im Team lernen.
\label{cha:fazit}
\section{Erkenntnisse}
Um die gesammelten Erfahrungen und Messungen später verwenden zu können ist eine genau Dokumentation derer nötig.
Wenn etwas nicht festgehalten wird ist es im nach hinein schwer nachzuvollziehen.
Bei größeren Projekten ist es wichtig, einen klaren Plan und eine klare Rollenverteilung zu haben.


\section{Schwierigkeiten und L\"osungsans\"atze}
\label{sec:probleme}
\subsection{Limitierung durch Gateway}
Da zwar zwei Nodes zur Verfügung standen, jedoch nur ein Gateway in Form eines USB Sticks konnte nur ein Entwicklungsteam gleichzeitig damit arbeiten.
Dies war zu Anfang des Projekts nicht problematisch da zunächst der Aufbau der Messboxen ansich geplant wurde.
Gegen Ende des Projekts hin führte dies jedoch zu Terminkonflikten was dazu führte,
dass die Teams teilweise Sondertermine vereinbaren mussten um Zugriff auf den USB Stick bekommen zu können wenn das andere Team nicht anwesend war.

\subsection{Gelieferte Konfiguration des VLINK200}
Wie bereits in Kapitel \ref{sec:theo} beschrieben haben wir am Anfang des Projekts sehr viel Zeit verloren weil wir versucht hatten, eine Konfiguration zu nutzen, die in der Hardware nicht umgesetzt war.
Dadurch ergab sich die Suche nach dem Problem an der falschen Stelle.
Die Tatsache, dass wir die DMS nicht direkt an den Node anschließen konnten führte zu einer viel höheren Komplexität des Projekts als ursprünglich angenommen.

\section{Ausblick}
Die grundsätzliche Funktion des Messdatenloggers ist mit den zuvor beschriebenen Testszenarien nachgewiesen.
In Zukunft können komplexere und mechanisch sinnvollere Versuche, einen echten Mehrwert zu der Entwicklung des FlyingSuits beitragen.
Denkbar wäre z.B. die Belegung der vier übrigen Analogen Eingänge mit Sensoren, wie z.B. Beschleunigungssensoren.
Zudem ermöglicht der VLink 200, den Betrieb zweier Datenlogger gleichzeitig, was die doppelte Anzahl an Messpunkten zur folge hätte.
Hierdurch könnten die in der Testbeschreibung angesprochenen segmentweisen Biegungs- und Torsionstests realisiert werden.
Darüber hinaus bleibt es das finale Ziel, Messwerte aus einem realen Flugversuch zu gewinnen.

Was das Fahrrad betrifft, so kann der entwickelte Aufbau in WPMs oder ähnlichen Veranstaltungen Verwendung finden.
Auch hier könnten weitere Sensoren, wie beispielsweise Beschleunigungssensoren angeschlossen werden.