\chapter{Fazit}
\label{cha:fazit}
\todo{}
\section{Erkenntnisse}

\section{Schwierigkeiten und L\"osungsans\"atze}
\label{sec:probleme}
\subsection{Limitierung durch Gateway}
Da zwar zwei Nodes zur Verfügung standen, jedoch nur ein Gateway in Form eines USB Sticks konnte nur ein Entwicklungsteam gleichzeitig damit arbeiten.
Dies war zu Anfang des Projekts nicht problematisch da zunächst der Aufbau der Messboxen ansich geplant wurde.
Gegen Ende des Projekts hin führte dies jedoch zu Terminkonflikten was dazu führte,
dass die Teams teilweise Sondertermine vereinbaren mussten um Zugriff auf den USB Stick bekommen zu können wenn das andere Team nicht anwesend war.

\subsection{gelieferte Konfiguration des VLINK200}\
\section{Optimierungsm\"oglichkeiten}\
\section{Ausblick}
Die grundsätzliche Funktion des Messdatenloggers ist mit den zuvor beschriebenen Testszenarien nachgewiesen.
In Zukunft können komplexere und mechanisch sinnvollere Versuche, einen echten Mehrwert zu der Entwicklung des FlyingSuits beitragen.
Denkbar wäre z.B. die Belegung der vier übrigen Analogen Eingänge mit Sensoren, wie z.B. Beschleunigungssensoren.
Zudem ermöglicht der VLink 200, den Betrieb zweier Datenlogger gleichzeitig, was die doppelte Anzahl an Messpunkten zur folge hätte.
Hierdurch könnten die in der Testbeschreibung angesprochenen segmentweisen Biegungs- und Torsionstests realisiert werden.
Darüber hinaus bleibt es das finale Ziel, Messwerte aus einem realen Flugversuch zu gewinnen.